\subsubsection{C7 Pregunta 6}
\begin{lstlisting}
 class Program
     {
        static int a = 5;
         static void Main(string[] args)
         {
             a = a +fun1();
             Console.WriteLine(a);
             Console.ReadLine();
        }
 
        static int fun1()
         {
             a = 17;
             return 3;
         }
     }
 \end{lstlisting}
 OUTPUT\\
 8\\ 
Como podemos ver en Si Sharp también se cumple la regla de la asociativad. En este caso el operador '+' asoicativdad desde la izquierda. Primero a=5, luego le suma 3; cuyo resultado que es 8 se le asigna a la variable a.\\
En cambio si fuese:  a = fun1()+a. En ese caso fun1() retorna 3 y actualiza la variable global 'a'=17 y luego suma 17+3 asignando 20 a la varaible a.
  