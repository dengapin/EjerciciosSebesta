\documentclass[12pt,oneside]{article}
\usepackage{geometry}                                % See geometry.pdf to learn the layout options. There are lots.
\usepackage{listings}				% Permite utilizar lenguajes de programacion dentro de latex
\geometry{a4paper}                                           % ... or a4paper or a5paper or ... 
%\geometry{landscape}                                % Activate for for rotated page geometry
%\usepackage[parfill]{parskip}                    % Activate to begin paragraphs with an empty line rather than an indent
\usepackage{graphicx}                                % Use pdf, png, jpg, or epsß with pdflatex; use eps in DVI mode
                                                                % TeX will automatically convert eps --> pdf in pdflatex                
\usepackage{amssymb}

\usepackage[spanish]{babel}                        % Permite que partes automáticas del documento aparezcan en castellano.
\usepackage[utf8]{inputenc}                        % Permite escribir tildes y otros caracteres directamente en el .tex
\usepackage[T1]{fontenc}                                % Asegura que el documento resultante use caracteres de una fuente apropiada.

\usepackage{hyperref}                                % Permite poner urls y links dentro del documento

\title{Ejercicios de Programación - Sebesta}
\author{Lenguajes de Programación - ESPOL}

%\date{}                                                        % Activate to display a given date or no date

\begin{document}
\maketitle

\section{Introducción}
Las respuestas propuestas en este repositorio son producto del trabajo de los estudiantes de la materia ``Lenguajes de Programación'' de la ESPOL, correspondientes a las preguntas del libro de Robert Sebesta, Concepts of Programming Languages.

\section{Preguntas y Respuestas}

\subsection{Capítulo 5: Nombres, Enlaces y Alcances.}
\subsubsection{C5 Pregunta 4: Subprogramas que definen variables que son utilizadas en otro subprograma}

\begin{verbatim}

def leer():
       listadeCompras=open('lista.txt','r')
       linea=listadeCompras.readline()
       while linea!="":
             linea=listadeCompras.readline()
             generarLista(linea)
             
def generarLista(linea):
                     if 'limon' or 'naranja' or 'kiwi' in linea:
                             lista1(linea)
                     elif 'almendras' or 'cacahuetes' or 'pimienta'  in linea:
                             lista2(linea)
                                      
def lista1(linea):
       print("Vitamina C:")
       for i in range(3):
             if i==1:
                    print(str(linea)+"sirve para los huesos")
             if i==2:
                    print(str(linea)+"sirve para los dientes")
             if i==3:
                    print(str(linea)+"reduce los sintomas de alergias")
               

def lista2(linea):
       print("Vitamina E:")
       for i in range(3):
             if i==1:
                    print(str(linea)+"sirve para la circulacion")
             if i==2:
                    print(str(linea)+"sirve para el cabello")
             if i==3:
                    print(str(linea)+"sirve para el colesterol")
                        
leer()

\end{verbatim}

\subsubsection{C5 Pregunta5: Generea la secuencia dada en C,C++ y java}

\begin{verbatim}

PROGRAMA EN C ---GENERA ERROR LA SECUENCIA PROPUESTA A COMPILAR

void main(void)
{
	x=21;
	printf("%i:",x);
	int x;
	x=42;
	printf("%i",x);
	Sleep(5000);
}

Lista de errores

Error	1	error C2065: 'x' : identificador no declarado
c:\users\dennise\documents\visual studio\2010\projects
\pruebadenisse\pruebadenisse.cpp	
10	1	pruebaDenisse

Error	2	error C2065: 'x' : identificador no declarado	
c:\users\dennise\documents\visual studio\2010\projects
\pruebadenisse\pruebadenisse.cpp
11	1	pruebaDenisse


PROGRAMA EN C++  ---GENERA ERROR LA SECUENCIA PROPUESTA A COMPILAR

void main(void)
{
	
	cout << "Ingresa 21:";
	cin >> x;
        int x;
	x=42;
	cout << "X vale:" << x;
	cin.get();cin.get();
	
}

Error	1	error C2065: 'x' : identificador no declarado 
c:\users\dennise\documents\visual studio 2010\projects
\pruebadenisse\pruebadenisse.cpp
13	1	pruebaDenisse


PROGRAMA EN JAVA ---GENERA ERROR LA SECUENCIA PROPUESTA A COMPILAR

public class PruebaDenisse {

    static void main(String[] args) {
        x=21;
        int x;
        x=42;
       System.out.println(x);
    }
}

Exception in thread "main" java.lang.RuntimeException: 
Uncompilable source code - cannot find symbol
  symbol:   variable x
  location: class pruebadenisse.NewMain 
	at pruebadenisse.NewMain.main(NewMain.java:17)
Java Result: 1


\end{verbatim}
En los tres lenguajes se debe especificar primero el tipo de la variable sin esto \\
resulta imposible gnerar el codigo planteado.
\subsubsection{C5 Pregunta6: Sentencia for alcance de la variabe declarada en ella en los tres lenguajes acontinuacion explicados }

\begin{verbatim}

JAVA
import java.io.*;
 
public class c5p6{
    public static void main(String args[]){
 	for(int i = 0; i< 10 ; i++)
 	    {
 		System.out.println(i);
 	    }
 	System.out.println(i);
     }
 }

C++
#include <iostream>
using namespace std;
 
int main(){
   for (int i = 0 ; i<10; i++){
     cout << i;
   }
   cout << i; 
   return 0;
 }

C#
using System;
 
 class c5p6
 {
   static void Main()
   {
     for(int i = 0 ; i < 10 ; i++)
       {
 	Console.WriteLine(i);
       }
     Console.WriteLine(i);
   }
 
 }

\end{verbatim}
En los tres lenguajes, el declarar una variable dentro del for\\
y luego intentar accederla fuera del bloque for, nos genera \\
un error. La variable tiene alcance local dentro del bloque for, y \\
solo puede ser accedida dentro del mismo.


\subsubsection{C5 Pregunta 7: Tres funciones en C donde se declare un arreglo de forma\\
estatica, otra stack y la ultima declaracion como heap}

\begin{verbatim}
STACK 

#define MAX 3000

int main (int argc, char *argv[])
{
 
  int arreglo[MAX];
  int i;
  srand(time(NULL));

  for(i=0;i<MAX-1;i++)
  arreglo[i]=1+rand()%100;

  for(i=0;i<MAX-1;i++){
  int doble;
  doble=arreglo[i]*2;
  printf("El doble del numero aleatorio en la posicion %d : %d\n",i,doble);
 
  }
  return 0;
}

HEAP


#define MAX 3000
int main (int argc, char *argv[])
{
 
  int arreglo[3000]; 
  int i;
  int *p;
  srand(time(NULL));  


  for(i=0;i<2999;i++){
  p= (int *)malloc(3000*sizeof(int));
  arreglo[i]=1+rand()%100;
  *p=arreglo[i];
  }
 

  for(i=0;i<2999;i++){
  int doble;
  p= (int *)malloc(3000*sizeof(int));
  *p=arreglo[i];
  doble=*p*2;
  printf("El doble del numero aleatorio en la posicion %d manejado por heap es: %d\n",i,doble);
  
  }

  free(p);
  return 0;
}

STATIC

int main (int argc, char *argv[])
{
 
  static int arreglo[3000];
  int i;
  srand(time(NULL));

  for(i=0;i<2999;i++)
  arreglo[i]=1+rand()%100;

  for(i=0;i<2999;i++){
  int doble;
  doble=arreglo[i]*2;
  printf("El doble del numero aleatorio en la posicion %d con arreglo estatico es: %d\n",i,doble);
 
  }
  return 0;
}

\end{verbatim}
Stack tiene un aceso mas rapido, el espacio es manejado por el CPU es limitado y no puede ser redefinido.\\
En el caso del heap nosotros necesitamos manejar la memoria, acceso mas lento y no es limitado\\
Static declaracion unica de una variable que mantiene su dimension a lo largo del tiempo de vida del programa.


\subsection{Capítulo 6}
\subsubsection{C6 Pregunta 1}
\begin{lstlisting}
int main (int argc, char *argv[])
{
	int i,div_int;
	float div_float,j=11.00;
	int arreglo[3];
	arreglo[2]=10;
	div_int=arreglo[2]/2;
	div_float=j/2;
	if(div_int!=div_float){
		printf("%i diferente %.2f",div_int,div_float);
		 Sleep(1000);
	}if(div_int==div_float){
	     printf("%i igual %.2f",div_int,div_float);
		 Sleep(1000);
	}
	
  return 0;
}
 \end{lstlisting}
La compatibilidad de tipos en C es muy variada, si comparamos como en el primer \\
if es claro que apesar de que los dos son numeros cinco los decimales de el segundo\\
valor hacen que la igualdad no se cumpla en caso de haber salido un 5.00 la igualdad \\
se cumplia ignorando que sean de diferentes tipos.Esto hace que cuando comparamos en \\
c el se encargue de algunas conversiones para llevar a caboo comparaciones en el \\
programa, en caso de que cambiemos a int el divfloat nos advertira de la perdida de datos \\
pero no nos generara error alguno.
  
\subsubsection{C6 Pregunta 2}

\begin{lstlisting}
double *multiplicarxdos (double *input) {
  double *twice;
   twice = (double*)malloc(sizeof(double));
  *twice = *input * 2.0;
  return twice;
}

int main (int argc, char *argv[])
{
  int *edad = (int *)malloc(sizeof(int));
  *edad = 23;
  double *salario = (double*)malloc(sizeof(double));
  *salario = 12345.67;
  double *miLista = (double*)malloc(3 * sizeof(double));
  miLista[0] = 1.2;
  miLista[1] = 2.3;
  miLista[2] = 3.4;

  double *twiceSalary = multiplicarxdos(salario);

  printf("El doble del salario es %.3f\n", *twiceSalary);
  Sleep(1000);

  free(edad);
  free(salario);
  free(miLista);
  free(twiceSalary);

  return 0;

}
 \end{lstlisting}
Se hace uso de la funcion free() en C cuando tenemos un acceso a memoria dinamica \\
en el Heap, en este caso como en el ejemplo debemos gestionar la memoria pidiendola \\
y despues liberandola repectivamente.
%\input{c6p7}
\subsection{Capítulo 7}
\subsubsection{C7 Pregunta 1}
\begin{lstlisting}

int fun(int *k) {
*k += 4;
return 3 * (*k) - 1;
}

void main() {
int i = 10, j = 10, sum1, sum2;
sum1 = (i / 2) + fun(&i);
sum2 = fun(&j) + (j / 2);
}

 \end{lstlisting}
En el programa la función fun() retorna el mismo valor para sum1 y sum2 la diferencia \\
esta en el orden al momento de hacer la suma. Sum 1 retorna 46 porque primero se \\
hace la división teniendo i como 10 pero en el segundo caso, Sum2 el valor de referencia
mandado en fun() cambio su direccion j por lo que da 48.Todo esta en el orden en la suma.

\subsubsection{C7 Pregunta 1}

\begin{lstlisting}

C++
int fun(int *k) {
*k += 4;
return 3 * (*k) - 1;
}

void main() {
int i = 10, j = 10, sum1, sum2;
sum1 = (i / 2) + fun(&i);
sum2 = fun(&j) + (j / 2);
}

 \end{lstlisting}

\begin{lstlisting}
package pruebadenisse;


JAVA
/**
 *
 * @author Dennise
 */
public class Main {

    public int fun(int k){
    k+=4;
    return 3*(k)-1;
    }
    
    public static void main(String[] args) {
        Main c=new Main();
        int i = 10, j = 10, sum1, sum2;
        sum1 = (i / 2) +c.fun(i);
        System.out.println(sum1);
        sum2 =  c.fun(j)+(j / 2);
        System.out.println(sum2);
    }
    
}

C#
namespace pruebaDennise
{
    class Programa
    {

        public int fun(int k)
        {
            k += 4;
            return 3 * (k) - 1;
        }
        static void Main(string[] args)
        {

            Programa p = new Programa();
             int i = 10, j = 10, sum1, sum2;
                sum1 = (i / 2) +p.fun(i);
                Console.WriteLine(sum1);
                sum2 =  p.fun(j)+(j / 2);
                Console.WriteLine(sum2);
                Console.ReadLine();
                }
    }
}


 \end{lstlisting}
En CS y Java se observo el mismo resultado ya que son programas orientados a objeto. \\
Devuelven 46 y 46 respectivamente.en el caso de C++ cambia debido a los punteros devuelve\\
46 y48 respectivamente.


\subsubsection{C7 Pregunta 1}

\begin{lstlisting}

def pagoSemestral():

       pagoxmateria=120
       materiasxtomar=int(input("Cuantas Materias va a tomar este Semestre:   \n"))
       promediocarrera=int(input("Cual es su promedio general:    \n"))
       
       if(9<=promediocarrera):
          valorxpagar=(pagoxmateria*materiasxtomar)/2
          print("Tu valor a pagar en es: $"+str(valorxpagar))
       elif(8<=promediocarrera<9):
          valorxpagar=(pagoxmateria*materiasxtomar)-(pagoxmateria*materiasxtomar)*(20/100)
          print("Tu valor a pagar es: $"+str(valorxpagar))
       else:
          valorxpagar=(pagoxmateria*materiasxtomar)
          print("Tu valor a pagar es: $"+str(valorxpagar))
             
pagoSemestral()

 \end{lstlisting}
En Python como en la mayoria de lenguajes existe la regla de precedencia izquierda a derecha \\
con excepcion de la exponenciacion que es lo contrario. En este ejemplo se calcula el valor a pagar \\
en dolares para registrarse en un semestre de una universidad X.

\subsubsection{C7 Pregunta 4}
\begin{lstlisting}
 public class JavaApplication2 {
     final static int num=5;
     static int a=5;
    
     public static void main(String[] args) {          
           a = fun1()+a; 
           System.out.println(a); 
     }
 
     static int fun1() {
             a = 17;
             return 3;
        }
 
 }
 
 \end{lstlisting}
 OUTPUT\\
 20\\
 Como podemos ver en el main en la 1era linea, fun1() retorna 3 y actualiza la variable global 'a'=17 y luego suma 17+3 asignando 20 a 'a'. Esto se debe a que en JAVA el operador '+' tiene asociatividad desde la izquierda.\\
 En cambio si fuese:  a = a + fun1();. En este caso primero 'a'=5 + fun1() que retorne 3; dando una asignación de 8 a la variable a.
\subsubsection{C7 Pregunta 5}
\begin{lstlisting}

 int fun1();
 
 extern int a = 10;
 void main(){
 	
 	a = fun1()+a;  // a = a+fun1();
 	printf("\%d", a);
 	getch();
 }
 
 int fun1() {
 	a = 17;
 	return 3;
 }
 \end{lstlisting}
Como podemos ver en el main en la 1era linea, comparado con JAVA Y Si Shard en vez de que fun1() retorne 3 y actualize la variable global 'a' a 17,y luego sumarlos y asignarle 20 a 'a'. En C++, sea a = fun1()+a  ó  a = a+fun1(), llamado de función en la izq o derecha del operador en este caso '+'; siempre al llamar fun1(), este va actualizar la variable global a =17 y luego va sumarle 3, asignando un valor de 20 a  la variable 'a'. 
\subsubsection{C7 Pregunta 6}
\begin{lstlisting}
 class Program
     {
        static int a = 5;
         static void Main(string[] args)
         {
             a = a +fun1();
             Console.WriteLine(a);
             Console.ReadLine();
        }
 
        static int fun1()
         {
             a = 17;
             return 3;
         }
     }
 \end{lstlisting}
 OUTPUT\\
 8\\ 
Como podemos ver en Si Sharp también se cumple la regla de la asociativad. En este caso el operador '+' asoicativdad desde la izquierda. Primero a=5, luego le suma 3; cuyo resultado que es 8 se le asigna a la variable a.\\
En cambio si fuese:  a = fun1()+a. En ese caso fun1() retorna 3 y actualiza la variable global 'a'=17 y luego suma 17+3 asignando 20 a la varaible a.
  
%\input{c7p9}
\subsection{Capítulo 8}
\subsubsection{C8 Pregunta 3}

\lstset{language=JAva} 
 \begin{lstlisting}

 
 public class chapterEightExThree {

     
     public static void main(String args[]){
         int j=0; int k=0;
         
         switch(k){
             case 1: j= 2 * k -1;
             case 2: j = 2 * k - 1;
             case 3: j = 3 * k + 1;
             case 4: j = 4 * k - 1;
             case 5: j = 3 * k + 1;
             case 6: j = k - 2;
             case 7: j = k - 2;
             case 8: j = k - 2;
             default: System.out.println("Fuera de rango");
                 
         
         }
         
     
     }
 }
 \end{lstlisting}
\subsubsection{C8 Pregunta 4}

\lstset{language=C}      
 \begin{lstlisting}
 #include <stdio.h>
 #include <string.h>
 
 main()
 {
 int i=0;
 int j=-3;
 int key=j+2;
 
 for(i=0;i<10;i++){
     if((key==3)||(key ==2)){
 		j--;
 	}else{
 		if(key==0){
 			j=j+2;
 		
 		}else{
 			j=0;
 			if(j>0){
 				break;
 			}else{
 				j=3-i;
 			}
 		}
 	}		
 }
 }
 
 \end{lstlisting}
\subsubsection{C8 Pregunta 5}

\lstset{language=JAva}       
 \begin{lstlisting}

 public class chapterEightExFive {
     
     public static void main(String args[]){
         boolean encontrado =false;
         int n=5; //suponemos la dimesion 5
         int[][] x = new int[n][n];
         
         for(int i=0;i<=n;i++){
             int acum=0;
             for(int j=0;j<=n;j++){
                 if(x[i][j]==0){
                     acum++;
                 }
                 if(acum==n && !encontrado){
                     System.out.println("Encontrad");
                     encontrado=true;
                 }
             }      
         }
         
     }
 }
 }
 \end{lstlisting}
 Aunque el codigo presentado en el programa es mas corto, con seguridad el mismo ejercicio transcrito a lenguaje java presenta mayor legibilidad
\subsection{Capítulo 9}
%\input{c9p1}
\subsubsection{C9 Pregunta 5}

 \lstset{language=JAva}    
 \begin{lstlisting}

 public class chapterNineExFive {
 	static int m =100;
 	public static int[][] static_matriz1 = new int[m][m];
 	public static int[][] static_matriz2 = new int[m][m];
 	public static int[][] static_resul = new int[m][m];
 	
 	
 	public chapterNineExFive(){
 
 	}
 	
 	public void matricesRandom(int[][] a, int[][] b){
 		for(int fil=0;fil<m;fil++){
 			for(int col=0;col<m;col++){
 				a[fil][col] = (int)(Math.random()*m)+1;
 				b[fil][col] = (int)(Math.random()*m)+1;
 				
 			}
 		}
 	}
 	
 	public void operacionesStatic(){
 		this.matricesRandom(static_matriz1, static_matriz2);
 		
 		 for(int i = 0; i < m; i++){
 	            for (int j = 0; j < m; j++){
 	                for (int k = 0; k < m; k++){
 	                    static_resul[i][j] += 
                             static_matriz1[i][k] * static_matriz2[k][j];
 	                }
 	            }
 	        }

 	}
 	
 	public void operacionesDinamic(){
 		int[][] dinamic_matriz1 = new int[m][m];
 		int[][] dinamic_matriz2 = new int[m][m];
 		int[][] dinamic_resul = new int[m][m];
 		
 		this.matricesRandom(dinamic_matriz1, dinamic_matriz2);
 		
 		 for(int i = 0; i < m; i++){
 	            for (int j = 0; j < m; j++){
 	                for (int k = 0; k < m; k++){
 	                	dinamic_resul[i][j] += 
                                  dinamic_matriz1[i][k] * dinamic_matriz2[k][j];
 	                }
 	            }
 	        }
 	
 	}
 
 
 	public static void main(String args[]){
             
 		long startTime = System.nanoTime();
 		new  chapterNineExFive().operacionesStatic();
 		long endTime = System.nanoTime();
 		long duration = endTime - startTime;
 		System.out.println("Tiempo Estatico: " + duration);
                
                 
                 long startTimeD = System.nanoTime();
 		new  chapterNineExFive().operacionesDinamic();
 		long endTimeD = System.nanoTime();
 		long durationD = endTimeD - startTimeD;
                 System.out.println("Tiempo Dinamico: " + durationD);
 	}
 }
 \end{lstlisting}
 El resultado de este operacion fue una media alrededor de 14,4u milis para la operacion statica y alrededor de  13,6u para la operacion dinamica. Aunque la diferencia es casi minima, y casi imperceptible el metodo dinamico resulto ser mas rapido que el estatico.

% Continuar con los siguientes capítulos y ejercicios:
% Ch6: 1, 2, 7
% Ch7: 1 - 6, 9
% Ch8: 3, 4, 5
% Ch9: 1, 5
% Recuerden que todos corresponden a las secciones de "Programming Exercises".

\end{document}
