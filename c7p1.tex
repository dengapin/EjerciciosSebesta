\subsubsection{C7 Pregunta 1}
\begin{lstlisting}

int fun(int *k) {
*k += 4;
return 3 * (*k) - 1;
}

void main() {
int i = 10, j = 10, sum1, sum2;
sum1 = (i / 2) + fun(&i);
sum2 = fun(&j) + (j / 2);
}

 \end{lstlisting}
En el programa la función fun() retorna el mismo valor para sum1 y sum2 la diferencia \\
esta en el orden al momento de hacer la suma. Sum 1 retorna 46 porque primero se \\
hace la división teniendo i como 10 pero en el segundo caso, Sum2 el valor de referencia
mandado en fun() cambio su direccion j por lo que da 48.Todo esta en el orden en la suma.
