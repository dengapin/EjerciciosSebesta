\subsubsection{C7 Pregunta 1}

\begin{lstlisting}

C++
int fun(int *k) {
*k += 4;
return 3 * (*k) - 1;
}

void main() {
int i = 10, j = 10, sum1, sum2;
sum1 = (i / 2) + fun(&i);
sum2 = fun(&j) + (j / 2);
}

 \end{lstlisting}

\begin{lstlisting}
package pruebadenisse;


JAVA
/**
 *
 * @author Dennise
 */
public class Main {

    public int fun(int k){
    k+=4;
    return 3*(k)-1;
    }
    
    public static void main(String[] args) {
        Main c=new Main();
        int i = 10, j = 10, sum1, sum2;
        sum1 = (i / 2) +c.fun(i);
        System.out.println(sum1);
        sum2 =  c.fun(j)+(j / 2);
        System.out.println(sum2);
    }
    
}

C#
namespace pruebaDennise
{
    class Programa
    {

        public int fun(int k)
        {
            k += 4;
            return 3 * (k) - 1;
        }
        static void Main(string[] args)
        {

            Programa p = new Programa();
             int i = 10, j = 10, sum1, sum2;
                sum1 = (i / 2) +p.fun(i);
                Console.WriteLine(sum1);
                sum2 =  p.fun(j)+(j / 2);
                Console.WriteLine(sum2);
                Console.ReadLine();
                }
    }
}


 \end{lstlisting}
En CS y Java se observo el mismo resultado ya que son programas orientados a objeto. \\
Devuelven 46 y 46 respectivamente.en el caso de C++ cambia debido a los punteros devuelve\\
46 y48 respectivamente.

