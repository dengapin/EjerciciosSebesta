\subsubsection{C7 Pregunta 1}

\begin{lstlisting}

def pagoSemestral():

       pagoxmateria=120
       materiasxtomar=int(input("Cuantas Materias va a tomar este Semestre:   \n"))
       promediocarrera=int(input("Cual es su promedio general:    \n"))
       
       if(9<=promediocarrera):
          valorxpagar=(pagoxmateria*materiasxtomar)/2
          print("Tu valor a pagar en es: $"+str(valorxpagar))
       elif(8<=promediocarrera<9):
          valorxpagar=(pagoxmateria*materiasxtomar)-(pagoxmateria*materiasxtomar)*(20/100)
          print("Tu valor a pagar es: $"+str(valorxpagar))
       else:
          valorxpagar=(pagoxmateria*materiasxtomar)
          print("Tu valor a pagar es: $"+str(valorxpagar))
             
pagoSemestral()

 \end{lstlisting}
En Python como en la mayoria de lenguajes existe la regla de precedencia izquierda a derecha \\
con excepcion de la exponenciacion que es lo contrario. En este ejemplo se calcula el valor a pagar \\
en dolares para registrarse en un semestre de una universidad X.
